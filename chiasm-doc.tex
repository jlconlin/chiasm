\documentclass{article}
\usepackage{chiasm}
\usepackage[margin=1in]{geometry}
\usepackage{xcolor}
\usepackage{listings}

% Code listing style
\lstset{
  basicstyle=\ttfamily\small,
  breaklines=true,
  frame=single,
  backgroundcolor=\color{gray!10}
}

\title{The \texttt{chiasm} Package\\
       \large Typesetting Chiastic Literary Structures in \LaTeX}
\author{Jeremy Lloyd Conlin\thanks{This package was developed in collaboration with AI tools.}}
\date{\today}

\begin{document}

\maketitle

\tableofcontents

\section{Introduction}

The \texttt{chiasm} package provides a specialized environment for typesetting chiastic structures in \LaTeX. A chiasmus (plural: chiasmi) is a literary device where concepts or phrases are presented in a sequence and then repeated in reverse order, creating a mirror or symmetric structure. This pattern is common in ancient literature, rhetoric, and religious texts.

\subsection{Features}

\begin{itemize}
\item Automatic labeling (A, B, C, ..., C', B', A')
\item Progressive "staircase" indentation
\item Support for nested chiastic structures with prefixed labels (e.g., B.a, B.b)
\item Replacement nesting that automatically skips intermediate labels
\item Pivot markers for center elements without prime partners
\item Customizable indentation step size
\end{itemize}

\section{Basic Usage}

\subsection{Simple Chiasm}

The most basic chiasm has items that descend and then ascend:

\begin{lstlisting}
\begin{chiasm}
  \item First concept
  \item Second concept
  \chiasmreturn
  \item Second returns
  \item First returns
\end{chiasm}
\end{lstlisting}

\noindent This produces:

\begin{chiasm}
  \item First concept
  \item Second concept
  \chiasmreturn
  \item Second returns
  \item First returns
\end{chiasm}

\subsection{The \texttt{\textbackslash chiasmreturn} Command}

The \verb|\chiasmreturn| command marks the transition from the descending phase to the ascending phase. After this command:
\begin{itemize}
\item Labels gain a prime mark (')
\item Labels appear in reverse order
\item Indentation reverses to match the descending phase
\end{itemize}

\section{Extended Example}

Here's a more complex chiasm showing progressive indentation:

\begin{lstlisting}
\begin{chiasm}
  \item Outermost concept
  \item Second level
  \item Third level
  \item Fourth level (deepest)
  \chiasmreturn
  \item Fourth returns
  \item Third returns
  \item Second returns
  \item Outermost returns
\end{chiasm}
\end{lstlisting}

\noindent Result:

\begin{chiasm}
  \item Outermost concept
  \item Second level
  \item Third level
  \item Fourth level (deepest)
  \chiasmreturn
  \item Fourth returns
  \item Third returns
  \item Second returns
  \item Outermost returns
\end{chiasm}

\section{Nested Chiasms}

\subsection{Regular Nesting}

You can nest chiasms within items. Nested labels are automatically prefixed with the parent label:

\begin{lstlisting}
\begin{chiasm}
  \item Outer A
  \item Outer B
    \begin{chiasm}
      \item Inner a (labeled B.a)
      \item Inner b (labeled B.b)
      \chiasmreturn
      \item Inner b' returns
      \item Inner a' returns
    \end{chiasm}
  \chiasmreturn
  \item Outer B' returns
  \item Outer A' returns
\end{chiasm}
\end{lstlisting}

\noindent Result:

\begin{chiasm}
  \item Outer A
  \item Outer B
    \begin{chiasm}
      \item Inner a (labeled B.a)
      \item Inner b (labeled B.b)
      \chiasmreturn
      \item Inner b' returns
      \item Inner a' returns
    \end{chiasm}
  \chiasmreturn
  \item Outer B' returns
  \item Outer A' returns
\end{chiasm}

\subsection{Replacement Nesting}

A special feature allows you to start a nested chiasm \emph{without} an \verb|\item| before it. This automatically creates the next label level and prefixes nested items accordingly, while skipping that label on the return:

\begin{lstlisting}
\begin{chiasm}
  \item A level
  \item B level
  \item C level
  % No \item here - nested chiasm "becomes" D
  \begin{chiasm}
    \item Nested a (labeled D.a)
    \item Nested b (labeled D.b)
    \chiasmreturn
    \item Nested b' returns
    \item Nested a' returns
  \end{chiasm}
  \chiasmreturn
  \item C' returns (D' is skipped)
  \item B' returns
  \item A' returns
\end{chiasm}
\end{lstlisting}

\noindent Result:

\begin{chiasm}
  \item A level
  \item B level
  \item C level
  \begin{chiasm}
    \item Nested a (labeled D.a)
    \item Nested b (labeled D.b)
    \chiasmreturn
    \item Nested b' returns
    \item Nested a' returns
  \end{chiasm}
  \chiasmreturn
  \item C' returns (D' is skipped)
  \item B' returns
  \item A' returns
\end{chiasm}

\section{Pivot Elements}

Sometimes a chiasm has a center element that serves as a pivot point without a corresponding prime element. Use \verb|\chiasmpivot| before the center item:

\begin{lstlisting}
\begin{chiasm}
  \item First element
  \item Second element
  \chiasmpivot
  \item CENTER PIVOT (no prime partner)
  \chiasmreturn
  \item Second returns
  \item First returns
\end{chiasm}
\end{lstlisting}

\noindent Result:

\begin{chiasm}
  \item First element
  \item Second element
  \chiasmpivot
  \item CENTER PIVOT (no prime partner)
  \chiasmreturn
  \item Second returns
  \item First returns
\end{chiasm}

\section{Package Options}

Load the package with optional arguments to customize indentation:

\begin{lstlisting}
\usepackage[smallindent]{chiasm}  % 0.8em per level
\usepackage{chiasm}               % 1.2em per level (default)
\usepackage[largeindent]{chiasm}  % 1.6em per level
\end{lstlisting}

\section{Real-World Example}

Here's an example from the Book of Helaman, showing a complex nested chiasm:

\begin{lstlisting}
\begin{chiasm}
  \item And behold, there was peace in all the land,
  \item insomuch that the Nephites did go into whatsoever 
        part of the land they would...
  \item And it came to pass that they became exceedingly 
        rich, both the Lamanites and the Nephites;
  \item and they did have an exceeding plenty of gold...
    \begin{chiasm}
      \item Now the land south
      \item was called Lehi,
      \item and the land north
      \item was called Mulek,
      \item which was after the son of Zedekiah;
      \chiasmreturn
      \item for the Lord
      \item did bring Mulek
      \item into the land north,
      \item and Lehi
      \item into the land south.
    \end{chiasm}
  \chiasmreturn
  \item And behold, there was all manner of gold...
  \item and thus they did become rich.
  \item They did raise grain in abundance...
  \item And thus the sixty and fourth year did pass away 
        in peace.
\end{chiasm}
\end{lstlisting}

\newpage
\noindent Result:

\begin{chiasm}
  \item And behold, there was peace in all the land,
  \item insomuch that the Nephites did go into whatsoever part of the land they would, whether among the Nephites or the Lamanites.
  \item And it came to pass that they became exceedingly rich, both the Lamanites and the Nephites;
  \item and they did have an exceeding plenty of gold, and of silver, and of all manner of precious metals, both in the land south and in the land north.
    \begin{chiasm}
      \item Now the land south
      \item was called Lehi,
      \item and the land north
      \item was called Mulek,
      \item which was after the son of Zedekiah;
      \chiasmreturn
      \item for the Lord
      \item did bring Mulek
      \item into the land north,
      \item and Lehi
      \item into the land south.
    \end{chiasm}
  \chiasmreturn
  \item And behold, there was all manner of gold in both these lands, and of silver, and of precious ore of every kind;
  \item and thus they did become rich.
  \item They did raise grain in abundance, both in the north and in the south; and they did flourish exceedingly.
  \item And thus the sixty and fourth year did pass away in peace.
\end{chiasm}

\section{Label Formatting}

Labels are formatted according to depth:
\begin{itemize}
\item Depth 1: Uppercase letters (A, B, C, ...)
\item Depth 2: Lowercase letters (a, b, c, ...)
\item Depth 3: Roman numerals (i, ii, iii, ...)
\item Depth 4+: Arabic numerals (1, 2, 3, ...)
\end{itemize}

Prime marks (') are added to labels on the ascending (return) phase.

\section{Technical Details}

\subsection{Implementation}

The package uses:
\begin{itemize}
\item \texttt{enumitem} for the underlying list structure
\item \texttt{xparse} for robust command definitions
\item Depth-specific counters (chiasmi, chiasmii, chiasmiii, etc.)
\item Dynamic indentation calculations based on position in the chiasm
\end{itemize}

\subsection{Limitations}

\begin{itemize}
\item Maximum nesting depth is 10 levels (can be extended if needed)
\item Labels beyond 26 items at depth 1 or 2 will overflow alphabetic sequences
\item Complex paragraph formatting within items may require manual adjustment
\end{itemize}

\section{License}

This package is distributed under the \LaTeX\ Project Public License v1.3c or later.

\end{document}
